\begin{frame}{Conclusion}
    \framesubtitle{Closing thoughts and future works}

    \begin{itemize}
        \item To support the decision on the therapy protocol for a given AML patient, specialists resort to a prognostic of outcomes according to the prediction of response to treatment.
        \item The current ELN risk stratification is divided into favorable, intermediate, and adverse;
        \item Despite being widely used, it is very conservative since most patients receive an intermediate risk classification, delaying treatment and possibly worsening the patients' condition.
        
    \end{itemize}

\end{frame}

\begin{frame}{Conclusion}
    \framesubtitle{Closing thoughts and future works}

The results obtained are promising and indicate that these models can support the decision about the most appropriate therapy protocol safely;

    \begin{itemize}
        %\item This study presented a careful data analysis and explainable machine-learning models trained using the well-known Explainable Boosting Machine technique;
    
        
        \item The prediction model trained with gene expression data performed best, and although a expensive feature to obtain, it provides the most assertive result.

        \item The results indicated that using a set of genetic features so far unknown in the AML literature significantly increased the prediction model's performance;

        \item For future work, we suggest collecting more data to keep the models updated regarding the disease variations over time. Furthermore, the biological role of the genes \textit{KIAA0141}, \textit{MICALL2}, \textit{PHF6}, and \textit{SLC92A} in the pathogenesis and progression of AML deserves functional studies in experimental models.

        
    \end{itemize}
    
\end{frame}

% According to the patient's outcome prediction, these models can support the decision about the most appropriate therapy protocol. .

%We showed that the prediction model trained with gene expression data performed best. In addition, the results indicated that using a set of genetic features hitherto unknown in the AML literature significantly increased the prediction model's performance. The finding of these genes has the potential to open new avenues of research toward better treatments and prognostic markers for AML.

%For future work, we suggest collecting more data to keep the models updated regarding the disease variations over time. Furthermore, the biological role of the genes \textit{KIAA0141}, \textit{MICALL2}, \textit{PHF6}, and \textit{SLC92A} in the pathogenesis and progression of AML deserves functional studies in experimental models.


