
\begin{frame}{Before}
File structure of this project:
\begin{block}{File Structure}
    \begin{itemize}
        \item beamerthemesrc \hspace{2pt} \% The theme folder, just leave it.
        \item images \hspace{2pt} \% Put your images here.
        \item header.tex \hspace{2pt} \% Put your packages and commands here.
        \item main.tex \hspace{2pt} \% Compile this main.tex file.
        \item contents \hspace{2pt} \% Contents contained in the main.tex
    \end{itemize}
\end{block}
\end{frame}

\begin{frame}[fragile]{Getting Started}
\framesubtitle{Selecting the SINTEF Theme}
To start working with \texttt{sintefbeamer}, start a \LaTeX\ document with the
preamble:
\begin{block}{Minimum SINTEF Beamer Document}
\verb|\documentclass{beamer}|\\
\verb|\usetheme{src/sintef}
\usefonttheme[onlymath]{serif}
\titlebackground*{beamerthemesrc/ufscar/background}
%-------------add your packages here-------------
\usepackage{amsfonts,amsmath,oldgerm}
\usepackage[utf8]{inputenc}
\pdfstringdefDisableCommands{\let\textsuperscript\relax}
\usepackage{graphicx,thumbpdf,wasysym,ucs,pgf,listings,multirow,xspace,amsmath,auto-pst-pdf,psfrag,pstricks,xcolor,colortbl}
\usepackage{caption}
\captionsetup[figure]{labelformat=empty}
\usepackage[font=scriptsize]{caption}
\usepackage{tabu}
\usepackage{amsmath}
\usepackage{svg}
\usepackage{relsize}
\usepackage{tikz}
\usetikzlibrary{
  fit,shapes.misc
}
\usepackage{svg} 
\usepackage{longtable}
\usepackage{booktabs}


%-------------add your commands here-------------
\newcommand{\hrefcol}[2]{\textcolor{cyan}{\href{#1}{#2}}}
\newcommand{\testcolor}[1]{\colorbox{#1}{\textcolor{#1}{test}}~\texttt{#1}}


\newcommand{\runninguthor}{J.M. Almeida et al.}


|\\
\verb|\begin{document}|\\
\verb|\begin{frame}{Hello, world!}|\\
\verb|\end{frame}|\\
\verb|\end{document}|\\
\end{block}
\end{frame}

\begin{frame}[fragile]{Title page}
To set a typical title page, you call some commands in the preamble:
\begin{block}{The Commands for the Title Page}
\begin{verbatim}
\title{Sample Title}
\subtitle{Sample subtitle}
\author{First Author, Second Author}
\date{\today} % Can also be (ab)used for conference name &c.
\end{verbatim}
\end{block}
You can then write out the title page with \verb|\maketitle|.

To set a \textbf{background image} use the \verb|\titlebackground| command 
before \verb|\maketitle|; its only argument is the name (or path) of a graphic 
file.

If you use the \textbf{starred version} \verb|\titlebackground*|, the image 
will be clipped to a split view on the right side of the title slide.

\end{frame}

\begin{frame}[fragile]{Writing a Simple Slide}
\framesubtitle{It's really easy!}
\begin{itemize}[<+->]
\item A typical slide has bulleted lists
\item These can be uncovered in sequence
\end{itemize}
\begin{block}{Code for a Page with an Itemised List}<+->
\begin{verbatim}
\begin{frame}{Writing a Simple Slide}
  \framesubtitle{It's really easy!}
  \begin{itemize}[<+->]
    \item A typical slide has bulleted lists
    \item These can be uncovered in sequence
  \end{itemize}\end{frame}
\end{verbatim}
\end{block}

\end{frame}
