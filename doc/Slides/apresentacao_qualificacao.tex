\documentclass[xcolor=table]{beamer}

\usepackage[english]{babel}
\usepackage[utf8]{inputenc}
\usepackage{hyperref}
\usepackage{graphicx}
\usepackage{amssymb,amsmath}
\usepackage{array}
\newcolumntype{H}{>{\setbox0=\hbox\bgroup}c<{\egroup}@{}} %usado para esconder uma coluna na tabela
\usepackage[most]{tcolorbox}
\usepackage{ulem}
\usepackage{fancybox}%caixas de texto
\usepackage{xcolor}%caixas de texto
\usetheme{Ufscar}
\useinnertheme{rounded}
\setbeamertemplate{navigation symbols}{} %tira os símbolos de navegaçao dos slides
\setbeamertemplate{}[page number] %contador de slides
\setbeamersize{text margin left=0.30cm,text margin right=0.30cm}
\setbeamertemplate{frametitle continuation}{} %para de enumerar os slides que tem [allowframebreaks]

\usepackage{epsfig,psfrag}
\usepackage{epsfig,graphicx}

\usepackage{subcaption}

\usepackage{longtable}

\usepackage{tabularx} %usado para ajustar a largura da tabela na página
\usepackage{ltablex} % usa tabularx de forma similar ao longtable

\usepackage[Algoritmo]{algorithm}% http://ctan.org/pkg/algorithms
\usepackage{algpseudocode}% http://ctan.org/pkg/algorithmicx %use [noend] para não aparecer escrito "end funtion"

%%%%%%%%%%%%%%%%%%%%%%%%%%%%%%%%%%%%%%%%%%%%%%%%%%%%%%%%%%
%%%% Configuração do pacote algpseudocode para português
%%%%%%%%%%%%%%%%%%%%%%%%%%%%%%%%%%%%%%%%%%%%%%%%%%%%%%%%%%
%% Declaracoes em Português
%\algrenewcommand\algorithmicend{\textbf{fim}}
%\algrenewcommand\algorithmicdo{\textbf{faça}}
%\algrenewcommand\algorithmicwhile{\textbf{enquanto}}
%\algrenewcommand\algorithmicfor{\textbf{para}}
%\algrenewcommand\algorithmicif{\textbf{se}}
%\algrenewcommand\algorithmicthen{\textbf{então}}
%\algrenewcommand\algorithmicelse{\textbf{senão}}
%\algrenewcommand\algorithmicreturn{\textbf{retorna}}
%\algrenewcommand\algorithmicfunction{\textbf{função}}
%
%% Rearranja os finais de cada estrutura
%\algrenewtext{EndWhile}{\algorithmicend\ \algorithmicwhile}
%\algrenewtext{EndFor}{\algorithmicend\ \algorithmicfor}
%\algrenewtext{EndIf}{\algorithmicend\ \algorithmicif}
%\algrenewtext{EndFunction}{\algorithmicend\ \algorithmicfunction}
%
%% O comando For, a seguir, retorna 'para #1 -- #2 até #3 faça'
%%\algnewcommand\algorithmicto{\textbf{até}}
%%\algrenewtext{For}[3]%
%%{\algorithmicfor\ #1 $\gets$ #2 \algorithmicto\ #3 \algorithmicdo}
%%%%%%%%%%%%%%%%%%%%%%%%%%%%%%%%%%%%%%%%%%%%%%%%%%%%%%%%%%%%%%%%%%%

%#############################################################33
%Terceiro Caption para longtable
\usepackage{forloop,longtable}
\newcounter{count}
\setcounter{count}{1}
\makeatletter
\newbox\LT@secondhead
\def\endsecondhead{\LT@end@hd@ft\LT@secondhead}
\def\LT@output{%
	\ifnum\outputpenalty <-\@Mi
	\ifnum\outputpenalty > -\LT@end@pen
	\LT@err{floats and marginpars not allowed in a longtable}\@ehc
	\else
	\setbox\z@\vbox{\unvbox\@cclv}%
	\ifdim \ht\LT@lastfoot>\ht\LT@foot
	\dimen@\pagegoal
	\advance\dimen@-\ht\LT@lastfoot
	\ifdim\dimen@<\ht\z@
	\setbox\@cclv\vbox{\unvbox\z@\copy\LT@foot\vss}%
	\@makecol
	\@outputpage
	\ifvoid\LT@secondhead
	\setbox\z@\vbox{\box\LT@head}%
	\else
	\setbox\z@\vbox{\box\LT@secondhead}%
	\fi
	\fi
	\fi
	\global\@colroom\@colht
	\global\vsize\@colht
	\vbox
	{\unvbox\z@\box\ifvoid\LT@lastfoot\LT@foot\else\LT@lastfoot\fi}%
	\fi
	\else
	\setbox\@cclv\vbox{\unvbox\@cclv\copy\LT@foot\vss}%
	\@makecol
	\@outputpage
	\global\vsize\@colroom
	\ifvoid\LT@secondhead
	\copy\LT@head\nobreak
	\else
	\box\LT@secondhead\nobreak
	\fi
	\fi}
\makeatother
%Fim: Terceiro Caption para longtable
%############################################################################################

\title[Exame de Qualificação]{Contrastive Learning for Automatic Field Boundary Detection}
\author[PPGCCS -- UFSCar]{Marciele de Menezes Bittencourt,Co-Orientador: Dr. Renato Moraes Silva, Orientador: Prof. Dr. Tiago A. Almeida}
 
\vfill
\date{\footnotesize}

%Rodapé com número de slide
\expandafter\def\expandafter\insertshorttitle\expandafter{%
  \hspace*{\fill}%
  \insertframenumber\,/\,\inserttotalframenumber}
 
%===== cor do rodapé=========
\definecolor{azulEscuro}{RGB}{50,50,255}
\setbeamercolor*{palette quaternary}{fg=white,bg=azulEscuro!30!black} %cor do rodapé %muda outras cores do template
%============================
 
\abovecaptionskip 0.01 \abovecaptionskip %espaço antes caption
\belowcaptionskip 0 \belowcaptionskip %espaço depois caption

\newcommand{\up}[1]{\raisebox{1.5ex}[0pt]{#1}}

\usepackage{multirow}%tabelas
\usepackage{rotating} %rotacionar texto na tabela

%%%%%%%%%%%%%%%%%%%%%%%%%%%%%%%%%%%%%%%%%%%%%%%%%%%%%%%%%%%%%%%%%
\setbeamertemplate{itemize subitem}{$\blacktriangleright$}
\setbeamertemplate{itemize subsubitem}{$\bullet$} %\diamond
%\setbeamertemplate{itemize item}[square] %altera o símbolo que inicia 
\setbeamertemplate{itemize subitem}[triangle] %altera o símbolo que inicia um subitem em uma lista
%\setbeamertemplate{itemize subsubitem}[circle] %altera o símbolo que inicia um subsubitem em uma lista
 %%%%%%%%%%%%%%%%%%%%%%%%%%%%%%%%%%%%%%%%%%%%%%%%%%%%%%%%%%%%%%%%%
 
 \graphicspath{%
 	{figs/}%
 }



\begin{document}

%=============================================
\begin{frame}
	\begin{center}

	\vspace{0.3in}

	\LARGE \textbf{Crop field detection using graph-based image segmentation and contrastive learning}

	\vspace{0.2in}

	{\Large \textbf{Eduardo Garcia do Nascimento}}

	\vspace{0.2in}
	{\normalsize \textbf{Advisor: Prof. Dr. Tiago A. Almeida}}

	\vspace{0.2in}
	{\small{ Graduate Program in Computer Science\\  [-0.1in]Federal University of São Carlos -- UFSCar, Sorocaba}}

	{\normalsize April / 2022}

	\vfill
	\vfill

	\end{center}
\end{frame}
%=============================================

%=============================================
%\begin{frame}{Summary}
%	
%	\begin{itemize}
%		\item Precision Agriculture
%		\item Boundary Detection of Crop Fields
%		\item Problem
%		\item Contrastive Learning
%		\item Hypotheses
%		\item Implementation Proposal
%		\item Work Timeline
%	\end{itemize}
%	
%\end{frame}
%=============================================

%=============================================
\begin{frame}{ \normalsize What is Precision Agriculture?}

	\begin{figure}[htb]
		\centering
		\includegraphics[height=6cm]{figs/precisionag.jpg}
		\label{fig:precisionAgriculture}
	\end{figure}
	
	~\flushright \tiny {Images extracted from John Deere website. Available at: \url{https://www.deere.com/}. Access on \today}
	
\end{frame}
%=============================================

%=============================================
\begin{frame}{ \normalsize Remote Sensing in Precision Agriculture}
		
	\begin{columns}
		\column{0.6\textwidth}
		    \begin{tcolorbox}[colback=yellow!5,colframe=yellow!75!black]
		    A \textit{crop field} is the smallest observation unit designed by a farm based on the topography and mechanization plan. 
	        \end{tcolorbox}
	        
    		\begin{figure}[htb]
    			\centering
    			\includegraphics[height=3.5cm]{figs/fieldanalyzer.png}
    			\label{fig:textFieldAnalyzer}
    		\end{figure}

        \column{0.4\textwidth}

    		\begin{figure}[htb]
    			\centering
    			\includegraphics[height=3cm]{figs/gps.png}
    			\label{fig:gps}
    		\end{figure}
    		
    		\begin{figure}[htb]
    			\centering
    			\includegraphics[height=3cm]{figs/display.png}
    			\label{fig:display}
    		\end{figure}

	\end{columns}
	~\flushright \tiny {Images extracted from John Deere website. Available at: \url{https://www.deere.com/}. Access on \today})
	
\end{frame}
%=============================================

%=============================================
\begin{frame}{Boundary detection of crop fields}
	
	\begin{columns}
	\column{0.5\textwidth}

		\begin{figure}[htb]
			\centering
			\includegraphics[height=4cm]{figs/field_nodetection.png}
			\label{fig:fieldNoDetection}
		\end{figure}
		\pause

	\column{0.5\textwidth}
		\begin{columns}
		\column{0.2\textwidth}
		
		\huge {$\Rightarrow$ }
		
		\column{0.8\textwidth}	
			\begin{figure}[htb]
				\centering
				\includegraphics[height=4cm]{figs/field_detected.png}
				\label{fig:fieldDetected}
			\end{figure}
		\end{columns}
	\end{columns}
	
	~\flushright \tiny \cite{waldner2021}
	
\end{frame}
%=============================================

%=============================================
\begin{frame}{Problem}
	
	\begin{tcolorbox}[colback=red!5,colframe=red!75!black]
		Manually drawing field boundaries is a time-consuming and error-prone task.
	\end{tcolorbox}
	
	\begin{tcolorbox}[colback=red!5,colframe=red!75!black]
		The state-of-the-art automatic methods use deep learning to detect field boundaries but they need a significant number of labeled samples to work appropriately.
	\end{tcolorbox}
	
	\begin{tcolorbox}[colback=red!5,colframe=red!75!black]
		This requirement makes the application of deep learning restricted to limited areas of the globe. 
	\end{tcolorbox}
		
\end{frame}
%=============================================

%=============================================
\begin{frame}{Contrastive Learning}
	
	\begin{columns}
		\column{0.6\textwidth}

		\begin{figure}[htb]
      			\includegraphics[height=5cm]{figs/contrastive_learning.png}
      			\label{fig:simCLR}
		\end{figure}
		~\flushright \tiny \cite{chen2020}
	
		\column{0.4\textwidth}
  
		\pause
 
		Steps: \\
		\begin{itemize}
			\item Data augmentation
			\item Encoding
			\item Loss minimization
		\end{itemize}
	    
		\bigskip
	
	\end{columns}

\end{frame}
%=============================================

%=============================================
\begin{frame}{Contrastive Loss}
	
	\centering
	
	\begin{equation}
	L_{(i,j)}=-\log{\frac{\exp{\left(\boldsymbol{z}_i\cdot\boldsymbol{z}_j/\tau\right)}}{\sum_{k=1}^{2N}\mathbf{1}_{i\neq k}\cdot\exp{\left(\boldsymbol{z}_i\cdot\boldsymbol{z}_k/\tau\right)}}}
	\end{equation}
	~\flushright \tiny \cite{chen2020})
\end{frame}
%=============================================

%=============================================
\begin{frame}{Hypotheses}

	\begin{tcolorbox}[colback=blue!5,colframe=blue!75!black]
		As contrastive learning based methods require much fewer labeled samples than state-of-the-art deep learning models, its employment to detect field boundaries can lead to competitive results with minimal accuracy loss, demanding just a few labeled samples.
	\end{tcolorbox}
		
\end{frame}
%=============================================

%=============================================
\begin{frame}\frametitle{Implementation Proposal} 
	
	\begin{itemize}
		
		\item Labeling
		\begin{itemize}
			\item Labeled data from boundaries manually drawn by John Deere customers.
			\item Alternatively, operation data can be used to estimate boundaries.
		\end{itemize}
        
        \item Remote sensing
		\begin{itemize}
			\item Landsat \& Sentinel I and II images.
			\item Other sources to be evaluated during the research.
		\end{itemize}
		
		\item Model
		\begin{itemize}
			\item Train model using few samples of high-quality labeled data.
			\item Graph-based segmentation.
		\end{itemize}
	\end{itemize}
	
\end{frame}
%=============================================

%=============================================
\begin{frame}{Implementation Proposal}
	
	\centering
	
\tikzstyle{block} = [draw, rectangle, 
    minimum height=3em, minimum width=6em, fill=orange, fill opacity=0.2, text opacity=1]

% The block diagram code is probably more verbose than necessary
\begin{tikzpicture}[auto, node distance=4cm]
    % We start by placing the blocks
    
    \node [block] (data) {1. Data Processing};
    \node [block, right of=data] (training) {2. Model Training};
    \node [block, right of=training] (segmentation) {3. Segmentation};
    % We draw an edge between the controller and system block to 
    % calculate the coordinate u. We need it to place the measurement block. 
    \draw [->](data) -- (training);
    \draw [->](training) -- (segmentation);
\end{tikzpicture}
\end{frame}
%=============================================

%=============================================
%\begin{frame}\frametitle{Implementation Proposal} 
%
%	\begin{figure}[htb]
%		\centering
%		\includegraphics[height=7cm]{Exame de Qualificação/figs/onepage.png}
%		\label{fig:Implementation}
%	\end{figure}
%	~\flushright \tiny~ {Source: the author}
%\end{frame}
%=============================================

%=============================================
\begin{frame}\frametitle{Implementation Proposal} 

	\begin{figure}[htb]
		\centering
		\includegraphics[width=12cm]{Exame de Qualificação/figs/1.png}
		\label{fig:data}
	\end{figure}
	~\flushright \tiny~ {Source: the author}
\end{frame}
%=============================================

%=============================================
\begin{frame}\frametitle{Implementation Proposal} 

	\begin{figure}[htb]
		\centering
		\includegraphics[width=12cm]{Exame de Qualificação/figs/2.png}
		\label{fig:model}
	\end{figure}
	~\flushright \tiny~ {Source: the author}
\end{frame}
%=============================================

%=============================================
\begin{frame}\frametitle{Implementation Proposal} 

	\begin{figure}[htb]
		\centering
		\includegraphics[width=12cm]{Exame de Qualificação/figs/3.png}
		\label{fig:segmentation}
	\end{figure}
	~\flushright \tiny~ {Source: the author}
\end{frame}
%=============================================


%=============================================
%\begin{frame}{SimCLR}
%	
%	\begin{columns}
%		\column{0.5\textwidth}
%   		\begin{figure}[htb]
%          			\includegraphics[height=4cm]{Exame de Qualificação/figs/simclr_chart.png}
%          			\label{fig:SimCLRChart}
%    		\end{figure}
%    		~\flushright \tiny \cite{chen2020}
%	
%		\column{0.5\textwidth}
%    		\begin{figure}[htb]
%          			%\includegraphics[height=4cm]{figs/simclr_table.png}
%          			\label{fig:SimCLRTable}
%    		\end{figure}
%    		~\flushright \tiny \cite{chen2020}
%	
%	\end{columns}
%	
%
%\end{frame}
%=============================================

%%=============================================
%\begin{frame}\frametitle{Downstream task} 
%
%	\begin{figure}[htb]
%		\centering
%		\includegraphics[height=4cm]{figs/semantic_segmentation.png}
%		\label{fig:SemanticSegmentation}
%	\end{figure}
%	~\flushright \tiny \cite{rieder2019}
%	
%\end{frame}
%=============================================

%=============================================
\begin{frame}{Method Validation}

	\begin{itemize}
		\item The validation will follow the metrics~\cite{sokolova2009} employed by related state-of-the-art work~\cite{waldner2021}.
	\end{itemize}		

	\begin{itemize}
	    \item Thematic accuracy
		\begin{itemize}
			\item Pixel or Region-based accuracies.
			\item Overall, user's and producer's accuracy.
			\item Matthew's correlation coefficient (MCC).
		\end{itemize}
		
		\item Geometric accuracy
		\begin{itemize}
			\item Boundary similarity.
			\item Location similarity.
			\item Over and Under segmentation rates.
		\end{itemize}
	\end{itemize}
		
\end{frame}
%=============================================

%=============================================

\begin{frame}[fragile]
\frametitle{Timeline}

\begin{center}
	\begin{tiny}
    \begin{longtable}{|c|c|c|c|c|c|c|c|c|c|c|c|c|c|}
    \hline
   & \multicolumn{4}{c|}{First year} & \multicolumn{4}{c|}{Second year} \\
    \cline{2-9}
    \up{Tasks} &1\textordmasculine T&2\textordmasculine T&3\textordmasculine T&4\textordmasculine T
	&1\textordmasculine T&2\textordmasculine T&3\textordmasculine T&4\textordmasculine T\\
    \hline
    \multicolumn{1}{|c|}{\begin{tabular}[c]{@{}c@{}c@{}}Bibliographical \\review\\{ } \end{tabular}}&\cellcolor[gray]{.1}&\cellcolor[gray]{.4}&\cellcolor[gray]{.8}&\cellcolor[gray]{.8}&\cellcolor[gray]{.8}&\cellcolor[gray]{.8}&\cellcolor[gray]{.8}&\cellcolor[gray]{.8}\\
    \hline
    \multicolumn{1}{|c|}{\begin{tabular}[c]{@{}c@{}c@{}}Extraction of remote sensing\\images and\\manually drawn boundaries \end{tabular}}&&\cellcolor[gray]{.4}&\cellcolor[gray]{.8}&\cellcolor[gray]{.8}&&&&\\   
    \hline
    \multicolumn{1}{|c|}{\begin{tabular}[c]{@{}c@{}c@{}}Study of SimCLR\\ and similar\\ contrastive learning techniques\end{tabular}}&&\cellcolor[gray]{.4}&\cellcolor[gray]{.8}&\cellcolor[gray]{.8}&&&&\\ 
    \hline
    \multicolumn{1}{|c|}{\begin{tabular}[c]{@{}c@{}c@{}}Modeling, build \\e planning of\\required adaptations\end{tabular}}&&&\cellcolor[gray]{.8}&\cellcolor[gray]{.8}\cellcolor[gray]{.8}&\cellcolor[gray]{.8}&&&\\
        \hline
    \multicolumn{1}{|c|}{\begin{tabular}[c]{@{}c@{}c@{}}Execution of the model\\and evaluation \\of first results\end{tabular}}&&&&&\cellcolor[gray]{.8}&\cellcolor[gray]{.8}&&\\
        \hline
    \multicolumn{1}{|c|}{\begin{tabular}[c]{@{}c@{}c@{}}Further experiments based on\\the evaluation of\\the first results\end{tabular}}&&&&&&\cellcolor[gray]{.8}&\cellcolor[gray]{.8}&\\
        \hline
    \multicolumn{1}{|c|}{\begin{tabular}[c]{@{}c@{}c@{}}Writing the dissertation and \\sharing the results\\{ } \end{tabular}}&&&&&&&\cellcolor[gray]{.8}&\cellcolor[gray]{.8}\\
    \hline

    \end{longtable}
	\end{tiny}
\end{center}
\end{frame}
%=============================================
	
%=============================================
\begin{frame}[allowframebreaks]{References}
\tiny
\bibliographystyle{sbc}
%\bibliographystyle{IEEEtranSN_portugues}
\bibliography{bibFiles/references.bib}
\end{frame}
%=============================================

%=============================================
\begin{frame}{}
	\centering
	\Huge Questions?
		
\end{frame}
%=============================================

\end{document}
